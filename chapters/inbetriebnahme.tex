\chapter{Inbetriebnahme des SCHUNK Powerball}
\label{sec:Inbetriebnahme}

\section{CAN Verbindung zwischen Steuereinheit und Arm}
\section{Installation der Steuerungssoftware}
In diesem Kapitel soll ein detailierter Überblick über die den Installationsprozess des CAN Treibers und der IPA CANopen Libary gegeben werden
\begin{itemize}
 	\item Für eine Installation mit ROS, siehe Kapitel: \ref{sec:ROSinstall}
    \item Für eine ROS unabhängige Installation, siehe Kapitel: \ref{sec:withoutROSinstall}

\end{itemize}
\subsection{Erforderliche Third-Party Software}
Um eine problemfreie Installation zu gewährleisten, muss einige Software auf dem Steuercomputer
installiert sein.
\begin{itemize}
	\item Ein C++ Compiler mit Unterstützung des C++11 Standards, z.B. \textit{gcc} 
	oder \textit{g++} Version 2.6 oder höher.\\
	\texttt{sudo apt-get install gcc}
	\item Die aktuelle Version des Versionierungstool \textit{git}\\
	\texttt{sudo apt-get install git}
	\item 

\end{itemize}


\subsection{Installation des CAN Treibers}
\subsection{Installation der IPA CANopen Libary}
\subsubsection{Manuelle ROS unabhängige Installation}
% \subsubsection{}