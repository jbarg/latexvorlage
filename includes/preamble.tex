% A. Dokumentenklasse
% ---------------------------------------------------------------------
\documentclass[%
	pdftex,%              PDFTex verwenden da wir ausschliesslich ein PDF erzeugen.
	a4paper,%             Wir verwenden A4 Papier.
	oneside,%             Einseitiger Druck.
	12pt,%                Grosse Schrift, besser geeignet für A4.
	halfparskip,%         Halbe Zeile Abstand zwischen Absätzen.
	%chapterprefix,%       Kapitel mit 'Kapitel' anschreiben.
	headsepline,%         Linie nach Kopfzeile.
	footsepline,%         Linie vor Fusszeile.
	bibtotocnumbered,%    Literaturverzeichnis im Inhaltsverzeichnis nummeriert einfügen.
	idxtotoc%             Index ins Inhaltsverzeichnis einfügen.
]{scrbook}

% Lokalisierung für deutschen Text
% \usepackage[ngerman]{babel}
\usepackage[ngerman]{babel}

% \usepackage[latin1]{inputenc} % Latin1 (ISO-8859-1)
\usepackage[utf8]{inputenc}
\usepackage[T1]{fontenc}


% paket für Index-Erstellung
\usepackage{makeidx}





% Paket für erweiterte Tabelleneigenschaften
\usepackage{array}


% Paket zum einbetten von PDF, PNG, Grafiken
\usepackage{graphicx}

% Farben an verschiedenen Stellen
\usepackage{color}


\usepackage[%
	pdftitle={Arbeitstitel},%                        Titel des PDF Dokuments.
	pdfauthor={Jon Barg},%              Autor des PDF Dokuments.
	pdfsubject={Thema der Arbeit},%                    Thema des PDF Dokuments.
	pdfcreator={MacTex, LaTeX with hyperref and KOMA-Script},% Erzeuger des PDF Dokuments.
	pdfkeywords={keywords},
	pdfpagemode=UseOutlines,%                                  Inhaltsverzeichnis anzeigen beim Öffnen
	pdfdisplaydoctitle=true,%                                  Dokumenttitel statt Dateiname anzeigen.
	pdflang=de%                                               Sprache des Dokuments.
]{hyperref}

% Verschieben des Ladens von einzelnen Programmiersprachen
\usepackage[savemem]{listings}



\usepackage{lmodern}
% \usepackage[scaled=.85]{luximono}
% B. Einstellungen
% ---------------------------------------------------------------------

%  1. Definieren von eigenen benannten Farben.
%     Für spätere Verwendung in dem Dokument, definieren wir einzelne
%     benannte Farben.
%
\definecolor{LinkColor}{rgb}{0,0,0.5}
\definecolor{ListingBackground}{rgb}{0.85,0.85,0.85}


%
%  2. KOMA-Script Option, Zeilenumbruch bei Bildbeschreibungen.
%
\setcapindent{1em}

%
%  3. Stil der Kopf- und Fusszeilen.
%     Wir aktivieren mit 'headings' laufende Seitentitel.
%
\pagestyle{headings}

%
%  4. Stil der Überschriften auf normale Schrift.
%     Wir verwenden für die Überschriften den selben Font wie für den Text.
%
\setkomafont{sectioning}{\normalfont\bfseries}       % Titel mit Normalschrift
\setkomafont{captionlabel}{\normalfont\bfseries}     % Fette Beschriftungen 
\setkomafont{pagehead}{\normalfont\itshape}          % Kursive Seitentitel
\setkomafont{descriptionlabel}{\normalfont\bfseries} % Fette Beschreibungstitel

%
%  5. Farbeinstellungen für die Links im PDF Dokument.
%
\hypersetup{%
	colorlinks=true,%        Aktivieren von farbigen Links im Dokument (keine Rahmen)
	linkcolor=LinkColor,%    Farbe festlegen.
	citecolor=LinkColor,%    Farbe festlegen.
	filecolor=LinkColor,%    Farbe festlegen.
	menucolor=LinkColor,%    Farbe festlegen.
	urlcolor=LinkColor,%     Farbe von URL's im Dokument.
	bookmarksnumbered=true%  Überschriftsnummerierung im PDF Inhalt anzeigen.
}

%
%  6. Einstellungen für das 'listings' Paket.
%
\lstloadlanguages{TeX} % TeX sprache laden, notwendig wegen option 'savemem'
\lstset{%
	language=[LaTeX]TeX,     % Sprache des Quellcodes ist TeX
	numbers=left,            % Zelennummern links
	stepnumber=1,            % Jede Zeile nummerieren.
	numbersep=5pt,           % 5pt Abstand zum Quellcode
	numberstyle=\tiny,       % Zeichengrösse 'tiny' für die Nummern.
	breaklines=true,         % Zeilen umbrechen wenn notwendig.
	breakautoindent=true,    % Nach dem Zeilenumbruch Zeile einrücken.
	postbreak=\space,        % Bei Leerzeichen umbrechen.
	tabsize=2,               % Tabulatorgrösse 2
	basicstyle=\ttfamily\footnotesize, % Nichtproportionale Schrift, klein für den Quellcode
	showspaces=false,        % Leerzeichen nicht anzeigen.
	showstringspaces=false,  % Leerzeichen auch in Strings ('') nicht anzeigen.
	extendedchars=true,      % Alle Zeichen vom Latin1 Zeichensatz anzeigen.
	backgroundcolor=\color{ListingBackground}} % Hintergrundfarbe des Quellcodes setzen.

%
% C. NEUE MAKROS UND UMGEBUNGEN
% ---------------------------------------------------------------------------


%
%  1. Umgebung für Änerungsliste mit einem speziellen Aufzählungszeichen.
%
\newenvironment{ListChanges}%
	{\begin{list}{$\diamondsuit$}{}}%
	{\end{list}}

%
%  2. Ersatz für die \LaTeX und \TeX Befehle für korrekte Darstellung.
%     Wir verwenden die 'Latin Modern Family' ('lm') als Font, da diese im
%     vergleich zu 'Computer Modern' ('cm') auch PostScript Dateien
%     anbieten, was zu einer schöneren Darstellung im PDF führt.
%
\newcommand{\DMLLaTeX}{{\fontfamily{lmr}\selectfont\LaTeX}}
\newcommand{\DMLTeX}{{\fontfamily{lmr}\selectfont\TeX}}

\def\AmS{$\mathcal{A}$\kern-.1667em\lower.5ex\hbox
    {$\mathcal{M}$}\kern-.125em$\mathcal{S}$}
\def\AmSmath{\AmS{}math}

%
% D. AKTIONEN
% ---------------------------------------------------------------------------
%

%
%  1. Index erzeugen.
%
\makeindex

%
% E. SILBENTRENNUNG
% ---------------------------------------------------------------------------
%

\hyphenation{De-zi-mal-trenn-zeichen In-stal-la-ti-ons-as-sis-tent}

%
% ===========================================================================
% EOF
%


